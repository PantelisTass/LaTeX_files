\documentclass{article}
% preamble.tex
\usepackage{amsmath, amsthm, amssymb}
\usepackage{mathtools}
\usepackage{mathabx}
\usepackage{adjustbox}
\usepackage{graphicx}% Required for inserting images
\usepackage{tikz-cd} 
\usepackage{tikz}
\usepackage{comment}
\usetikzlibrary{calc, graphs, positioning, backgrounds, decorations.pathmorphing}
\usepackage{forest}
\usepackage{pgfplots, pgfplotstable}
\usepackage{indentfirst}
\pgfplotsset{compat=1.17}
\usepackage[a4paper,width=150mm,top=25mm,bottom=25mm]{geometry}
\usepackage{fancyhdr}
\pagestyle{fancy}
\usepackage{ifthen}
\usepackage{float}
\usepackage{subcaption}
\usepackage{xkeyval} % Load xkeyval first
\usepackage[colorlinks=true,linkcolor=blue]{hyperref} % Load hyperref with simplified options
\usepackage[dvipsnames]{xcolor}
\usepackage{tcolorbox}
\usepackage{wrapfig}
\usepackage[framemethod=tikz]{mdframed}
\usepackage{atbegshi}% http://ctan.org/pkg/atbegshi


\definecolor{block-gray}{gray}{0.95}
\newtcolorbox{examplesblock}[2][]{%
    grow to right by=-0mm,
    grow to left by=-0mm, 
    opacityback=0,
    frame hidden, boxrule=0pt,
    boxsep=0pt,
    breakable,
    enhanced jigsaw,
    borderline west={2pt}{0pt}{black},
    title={#2\par},
    colbacktitle={black},
    coltitle={black},
    fonttitle={\bfseries},
    attach title to upper={},
    #1,
}

\definecolor{}{gray}{0.95}
\newtcolorbox{codeblock}[2][]{%
    grow to right by=-0mm,
    grow to left by=-0mm, 
    opacityback=0,
    frame hidden, boxrule=0pt,
    boxsep=0pt,
    breakable,
    enhanced jigsaw,
    borderline west={2pt}{0pt}{black},
    title={#2\par},
    colbacktitle={black},
    coltitle={black},
    fonttitle={\large\bfseries},
    attach title to upper={},
}

\titleformat{\chapter}[display]   
{\normalfont\huge\bfseries}{\chaptertitlename\ \thechapter}{20pt}{\Huge}   
\titlespacing*{\chapter}{0pt}{-30pt}{40pt}

\newtheoremstyle{mystyle}%                % Name
  {1pt}%                                     % Space above
  {}%                                     % Space below
  {\normalfont}%                                     % Body font
  {}%                                     % Indent amount
  {\bfseries}%                            % Theorem head font
  {.}%                                    % Punctuation after theorem head
  { }%                                    % Space after theorem head, ' ', or \newline
  {\thmname{#1}\thmnumber{ #2}\thmnote{ (#3)}}%                                     % Theorem head spec (can be left empty, meaning `normal')

  \theoremstyle{mystyle}
\newtheorem*{remark}{Remark}

\NewDocumentCommand{\newframedtheorem}{O{}momo}{%
  \IfNoValueTF{#3}
   {%
    \IfNoValueTF{#5}
     {\newtheorem{#2}{#4}}
     {\newtheorem{#2}{#4}[#5]}%
    }
   {\newfratheorem{#2}[#3]{#4}}
  \tcolorboxenvironment{#2}{#1}%
}
\newmdtheoremenv{theorem}{Theorem}[section]


\newtheorem{corollary}{Corollary}[section]
\newmdtheoremenv{boxcor}{Corollary}[theorem]
\newtheorem{lemma}[theorem]{Lemma}
\newtheorem{prop}[section]{Proposition}
\newtheorem{definition}{Definition}[section] 
%\newtheorem{theorem}{Theorem}[section] 

\newmdtheoremenv{boxdef}{Definition}[section]
\newmdtheoremenv{boxlemma}{Lemma}[section]
\newmdtheoremenv{boxprop}{Proposition}[section]

\newcommand{\here}[2]{\tikz[remember picture]{\node[inner sep=0](#2){#1}}}

\newcommand{\N}{\ensuremath{\mathbb{N}}}
\newcommand{\R}{\ensuremath{\mathbb{R}}}
\newcommand{\Z}{\ensuremath{\mathbb{Z}}}
\newcommand{\Q}{\ensuremath{\mathbb{Q}}}
\newcommand{\C}{\ensuremath{\mathbb{C}}}
\newcommand{\PP}{\ensuremath{\mathbb{P}}}
\newcommand{\F}{\mathcal{F}}

\newcommand{\Real}{\operatorname{Re}}
\newcommand{\Img}{\operatorname{Im}}

\DeclareMathOperator{\spn}{span}
\DeclareMathOperator{\Id}{Id}
\DeclareMathOperator{\essup}{essup}
\DeclareMathOperator{\scrC}{\mathcal{C}}
\DeclareMathOperator{\scrM}{\mathcal{M}}
\DeclareMathOperator{\supp}{supp}
\DeclareMathOperator{\intr}{int}
\DeclareMathOperator{\conv}{conv}
\DeclareMathOperator{\diam}{diam}
\DeclareMathOperator{\ext}{Ext}
\DeclareMathOperator{\dist}{dist}


\newcommand{\inmath}[2]{$#1\in#2$}
\newcommand{\allin}[2]{for all $#1\in#2$}
\newcommand{\isthere}{there exists }

\newcommand{\vectsp}[1]{\mathbf{#1}}
\newcommand{\norm}[1]{\left\lVert#1\right\rVert}
\newcommand{\unitball}[1]{B_{#1}}
\newcommand{\intunitball}[1]{D_{#1}}
\newcommand{\unitsphere}[1]{S_{#1}}

\newcommand{\bigslant}[2]{{\raisebox{.2em}{$#1$}\left/\raisebox{-.2em}{$#2$}\right.}}
\newcommand{\bracket}[2]{\langle #1, #2\rangle}
 % Method of Enrico Gregorio ("egreg")
 % https://tex.stackexchange.com/a/22255/13492:
 \newcommand\restrict[2]{% make the whole thing an ordinary symbol
   \left.\kern-\nulldelimiterspace % automatically resize the bar with \right
   #1% the function
   \littletaller % pretend it's a little taller at normal size
   \right|_{#2}%
   }
 \newcommand{\littletaller}{\mathchoice{\vphantom{\big|}}{}{}{}}

 % Method of Heiko Oberdiek
 % https://tex.stackexchange.com/a/67233/13492
 \makeatletter
 \newcommand*{\scaleddelims}[3]{%
   \ensuremath{%
     \mathpalette{\@scaleddelims{#1}{#2}}{#3}%
   }%
 }   
 \newcommand*{\@scaleddelims}[4]{%
   % #1: left delimiter
   % #2: right delimiter
   % #3: \displaystyle, \textstyle, ...
   % #4: inner formula
   \begingroup
     #3%
     \sbox0{$\m@th#3\vphantom{A}#4$}%
     \setbox2\vbox{\hbox{$\m@th#3#1$}\kern\z@}%
     \setbox4\vbox{\hbox{$\m@th#3#2$}\kern\z@}%
     \setbox6\hbox{$#3\vcenter{}$}%
     \ifx\downharpoonleft#1\relax  
       \let\DelimLeft=L%
     \else\ifx\upharpoonleft#1%
       \let\DelimLeft=L%
     \else\ifx\downharpoonright#1%
       \let\DelimLeft=R%
     \else\ifx\upharpoonright#1%
       \let\DelimLeft=R%
     \fi\fi\fi\fi
     \ifx\downharpoonleft#2\relax
       \let\DelimRight=L%
     \else\ifx\upharpoonleft#2\relax
       \let\DelimRight=L%
     \else\ifx\downharpoonright#2\relax
       \let\DelimRight=R%
     \else\ifx\upharpoonright#2\relax
       \let\DelimRight=R%
     \fi\fi\fi\fi
     \ifx\DelimLeft L%
       \wd2=.6\wd2
     \fi
     \ifx\DelimRight L%
       \wd4=.6\wd4
     \fi
     \ifx\DelimLeft R%
       \sbox2{\kern-.4\wd2\box2}%
     \fi
     \ifx\DelimRight R%
       \sbox4{\kern-.4\wd4\box4}%
     \fi
     \dimen0=\ht0 %
     \advance\dimen0 by -\ht6 %
     \dimen2=\dp0 %
     \advance\dimen2 by \ht6 %
     \ifdim\dimen2>\dimen0 %  
       \dimen0=\dimen2 %
     \else
       \dimen0=\dimen0 %
     \fi
     \dimen2=\ht6 %
     \advance\dimen2 by -\dimen0 %
     \dimen0=2\dimen0 %
     \def\DelimCorr{%  
       \mskip.5\thinmuskip
       \nonscript\mskip.5\thinmuskip
     }%
     \mathopen{%
       \ifx\DelimLeft R\DelimCorr\fi
       \raisebox{\dimen2}{\resizebox{!}{\dimen0}{\box2}}%
       \ifx\DelimLeft L\DelimCorr\fi
     }%
     \begingroup
       #3#4%
     \endgroup
     \mathclose{%
       \ifx\DelimRight R\DelimCorr\fi
       \raisebox{\dimen2}{\resizebox{!}{\dimen0}{\box4}}%
       \ifx\DelimRight L\DelimCorr\fi
     }%
   \endgroup
 }\makeatother
 %

 \newcommand{\restr}[2]{#1\scaleddelims{\kern-0.5\nulldelimiterspace\upharpoonright}{\vphantom{.}}{_{#2}}}
\newcommand{\diff}{\ensuremath{\operatorname{d}\!}}

  % Include the preamble from an external file
\usepackage{algorithm}
\usepackage{algpseudocode}


\definecolor{}{gray}{0.95}
\newtcolorbox{codeblock}[2][]{%
    grow to right by=-0mm,
    grow to left by=-0mm, 
    opacityback=0,
    frame hidden, boxrule=0pt,
    boxsep=0pt,
    breakable,
    enhanced jigsaw,
    borderline west={2pt}{0pt}{black},
    title={#2\par},
    colbacktitle={black},
    coltitle={black},
    fonttitle={\large\bfseries},
    attach title to upper={},
}




%\AtBeginDocument{\AtBeginShipoutNext{\AtBeginShipoutDiscard}\addtocounter{page}{-1}}
\fancyhead{}
\fancyhead[L]{\text{Imperial UROP 2022}}
\fancyhead[R]{\text{Pantelis Tassopoulos}}

\title{\Huge Random constructions in the plane \\ 
\huge Imperial College summer research supervised by Dr. Davoud Cheraghi}
\author{\Large Pantelis Tassopoulos}
\date{\Large Summer 2022}

\begin{document}


\maketitle

\newpage 
\tableofcontents

\newpage 
\section{Overview}

More specifically, I studied the notion of harmonic measure in the plane, its various formulations involving conformal maps and Brownian motion, culminating in the study of the so-called conformally balanced trees following work from Professor of mathematics at Stony Brook Christopher Bishop. This was a very profitable experience as it helped me further refine my analytical problem-solving and decomposition skills due to the nature of the work in the project. In conjunction with the above, my my communication and organisational skills were invariably improved as I engaged in weekly meetings with my supervisor Dr. Cheraghi, wherein I discussed the progress of the project and received feedback on approaches to obstacles, incorporating said suggestions into the project. I obtained a lot of insight into the world of academia and the way research is conducted. 

\newpage
\section{Conformally Balanced Trees}
\begin{itemize}
	\item mention paper of Chris Bishop 
	\item mention reading books, lecture notes, geometric function theory etc.
	\item Conjecture regarding Hausdorff distance and tree like structure
		(Conformally balanced trees have a unique embedding into the plane after undergoing hydrodynamic stabilisation --> explain maybe)
	\item mention zipper algorithm and what it does and how it was used to test the above hypothesis in a simple fashion
	
\end{itemize}


\begin{codeblock}{Degree three vertex Tree generation algorithm}
\begin{lstlisting}[language=Python]
import numpy as np
#Nesterov Potential 
n = 256
x = np.linspace(-1.5, 1.2, n)
y = np.linspace(-0.2, 2, n)
X, Y = np.meshgrid(x, y)

intervals = np.arange(1, 1e5, 20)

ntraj = 1000
# Initialize holder for trajectories
colors = plt.cm.jet(np.linspace(0,1,np.minimum(ntraj, 7)))
minima_nesterov = []
for j in tqdm(range(ntraj)):
    points_x, points_y = train_nesterov(intervals,
    learning_rate = 1e-4, a = 1, tolerance = 1e-5)
    minima_nesterov.append(MB_potential(points_x[-1],points_y[-1]))
    if j <=6: 
        plt.scatter(points_x, points_y, color = colors[j], s = 0.1)
        for i in range(len(points_x)-1):
            plt.annotate('', xy =[points_x[i+1],
	    points_y[i+1]], xytext= [points_x[i], points_y[i]],
                         arrowprops={'arrowstyle': '->', 'color': colors[j],  'lw': 1},
                         va='center', ha='center')

plt.contour(X, Y, vMB_potential(X, Y).clip(max=200), 8,
alpha=.75, cmap='viridis')
C = plt.contour(X, Y, vMB_potential(X,Y).clip(max=200), 8)
plt.title('Muller Brown potential with Nesterov accelerated GD')
plt.clabel(C, inline=1, fontsize=10)
plt.xticks([])
plt.yticks([])
#plt.legend()    

\end{lstlisting}
    
\end{codeblock}


\begin{codeblock}{Line Perturbation algorithm}
\begin{lstlisting}[language=Python]
import numpy as np
#Line perturbation (10 vertices)

import random
import numpy as np
from scipy.spatial import ConvexHull
from scipy.spatial.distance import directed_hausdorff
import matplotlib.pyplot as plt


#to contain file names
#files = ['_base', '_line1', '_line2', '_line3', '_line4', '_line5', '_line6', '_line7', '_line8', '_line9']
Hausdorff_distance = [] 
Points = []
Polygons = []

f = np.loadtxt("/Users/pantelistassopoulos/Downloads/Line/Points_10/verticeszTreeLine10Base.txt", usecols =(0,1), dtype = str)
x = f[1:-1,0]
y = f[1:-1,1]
points = []

for i in range(len(x)):
    points.append([x[i], y[i]])

points = np.matrix(points)  
Points.append(points)

plt.plot(points[:,0], points[:,1], 'r.', markersize = 1)

Polygon = []
for _ in range(len(points)-1):
    t = np.linspace(0,1,10)
    p1 = [points[_, 0], points[_, 1]]
    p2 = [points[_, 0], points[_+1, 1]]
    for _ in t:
        p = [float(p1[0])*(1-_) + float(p2[0])*_, float(p1[1])*(1-_) + float(p2[1])*_]
        Polygon.append(p)
Polygons.append(Polygon)
plt.plot(points[:,0], points[:,1], 'r-', markersize = 1)

for _ in range(10):
    f = np.loadtxt("/Users/pantelistassopoulos/Downloads/Line/Points_10/verticeszTreeLine10_"+str(_)+"_"+str(10-_)+".txt", usecols =(0,1), dtype = str)
    x = f[1:-1,0]
    y = f[1:-1,1]
    points = []
                   
    for i in range(len(x)):
        points.append([x[i], y[i]])
                   
    points = np.matrix(points)  
    Points.append(points)
            
    hull = ConvexHull(points)

    plt.plot(points[:,0], points[:,1], 'r.', markersize = 2)

    r = random.random()

    b = random.random()

    g = random.random()

    color = (r, g, b)

    for simplex in hull.simplices:
        plt.plot(points[simplex,0], points[simplex,1], 'b-')

    Polygon = []
    for simplex in hull.simplices:
        t = np.linspace(0,1,10)
        p1 = [points[simplex[0], 0], points[simplex[0], 1]]
        p2 = [points[simplex[1], 0], points[simplex[1], 1]]
        for _ in t:
            p = [float(p1[0])*(1-_) + float(p2[0])*_, float(p1[1])*(1-_) + float(p2[1])*_]
            Polygon.append(p)
    Polygons.append(Polygon)
        
        
    
plt.show()
for j in range(1,len(Polygons)):
    
    Hausdorff_distance.append(directed_hausdorff(Points[0], Points[j])[0])
    
plt.plot(list(range(1,len(Polygons))), Hausdorff_distance)
Hausdorff_distance
\end{lstlisting}
    
\end{codeblock}

\begin{codeblock}{Degree three vertex Tree Hausdorff distance}
\begin{lstlisting}[language=Python]
import numpy as np
import random
import numpy as np
from scipy.spatial import ConvexHull
from scipy.spatial.distance import directed_hausdorff
from scipy.optimize import curve_fit
import matplotlib.pyplot as plt
import pandas


def func_powerlaw(x, m, c, c0):
    return c0 + x**m * c

def logistic(x, a, b, c, d):
    return float(a) / (1.0 + np.exp(-float(c) * (x - float(d)))) + float(b)


def rational(x):
    return x/(1+np.abs(x))

def rational2(x, a, b, c, d):
    return a*rational(b*(x-c))+d

#to contain file names
files = [0,1,2,3,4,5,6,7,8,9,10,11,12,13, 14]
Hausdorff_distance = [] 
Points = []
Polygons = []

file = open("/Users/pantelistassopoulos/Downloads/Line/ThreeVertex/combined.txt", "w")

for _ in range(len(files)):
    f = np.loadtxt("/Users/pantelistassopoulos/Downloads/Line/ThreeVertex/verticeszThreeVertexDepth"+str(files[_])+".txt", usecols =(0,1), dtype = str)
    f1 = open("/Users/pantelistassopoulos/Downloads/Line/ThreeVertex/verticeszThreeVertexDepth"+str(files[_])+".txt", 'r')
    for line in f1:
            file.write(line)
    f1.close()
    x = f[1:-1,0]
    y = f[1:-1,1]
    points = []
                   
    for i in range(len(x)):
        points.append([x[i], y[i]])
                   
    points = np.matrix(points)  
                   
    hull = ConvexHull(points)
                   
    plt.plot(points[:,0], points[:,1], 'r.', markersize = 1)
    
    r = random.random()

    b = random.random()

    g = random.random()

    color = (r, g, b)
                   
    for simplex in hull.simplices:
        plt.plot(points[simplex,0], points[simplex,1], 'b-')
    
    Polygon = []
    for simplex in hull.simplices:
        t = np.linspace(0,1,10)
        p1 = [points[simplex[0], 0], points[simplex[0], 1]]
        p2 = [points[simplex[1], 0], points[simplex[1], 1]]
        for _ in t:
            p = [float(p1[0])*(1-_) + float(p2[0])*_, float(p1[1])*(1-_) + float(p2[1])*_]
            Polygon.append(p)
    Polygons.append(Polygon)
    Points.append(points)
    
circle = plt.Circle((0, 0), 2, color='g')
plt.gca().add_patch(circle)
plt.gca().set_aspect('equal', adjustable='box')

plt.show()

file.close()

for j in range(len(Points)):
    
    Hausdorff_distance.append(directed_hausdorff(Points[0], Points[j])[0])
    
#plt.plot(list(range(len(Points))), Hausdorff_distance, 'g+', markersize = 10)
print(Hausdorff_distance)

target_func = rational2
target_func2 = logistic


x = np.array(range(len(Points)))
y = np.array(Hausdorff_distance)
popt, pcov = curve_fit(target_func, x, y, maxfev = 2000)
X = np.linspace(0, len(Points))
#plt.plot(X, target_func(X, *popt), 'r--', label='rational', markersize = 1)
popt2, pcov2 = curve_fit(target_func2, x, y, maxfev = 2000)
#plt.plot(X, target_func2(X, *popt2), 'b--', label='logistic', markersize = 1)

fig, (ax1, ax2) = plt.subplots(1, 2)
fig.suptitle('Hausdorff Distance Regression Three Vertex')
ax1.plot(X, target_func(X, *popt), 'r--', label='rational', markersize = 1)
ax1.plot(list(range(len(Points))), Hausdorff_distance, 'g+', markersize = 10)
ax2.plot(X, target_func2(X, *popt2), 'b--', label='logistic', markersize = 1)
ax2.plot(list(range(len(Points))), Hausdorff_distance, 'g+', markersize = 10)
ax1.legend()
ax2.legend()
popt
\end{lstlisting}
    
\end{codeblock}


\begin{codeblock}{Tree insert edge left right}
\begin{lstlisting}[language=Python]
# add line segment left/ right

#Degree Three Vertex Tree Generation
def line(l, r):
    
    L = [1,2,3,4,5,6]
    L2 = [2,1,4,3,6,5]
    
    l_vertex = 3
    r_vertex = 5
    
    for _ in range(l):
        for α in range(len(L)):
            if L[α] > l_vertex:
                L[α] += 2
        for α in range(len(L2)):
            
            if L2[α] > l_vertex:
                L2[α] += 2
        L.append(l_vertex+1)
        L.append(l_vertex+2)
        L2.append(l_vertex+2)
        L2.append(l_vertex+1)
        l_vertex += 1
        r_vertex += 2
        
    
    for _ in range(r):
        for α in range(len(L)):
            if L[α] > r_vertex:
                L[α] += 2
        for α in range(len(L2)):
            
            if L2[α] > r_vertex:
                L2[α] += 2
        L.append(r_vertex+1)
        L.append(r_vertex+2)
        L2.append(r_vertex+2)
        L2.append(r_vertex+1)
        r_vertex += 1


    lines = []
    
    zipped_lists = zip(L, L2)

    sorted_pairs = sorted(zipped_lists)

    tuples = zip(*sorted_pairs)

    list1, list2 = [ list(tuple) for tuple in  tuples]
    for i in range(len(list1)):
        lines.append(str(list1[i])+" "+str(list2[i]))
        
    return lines
    

def lines(N):
    for n in range(N):
        f = open('/Users/pantelistassopoulos/Downloads/Line/Points_50/TreeLine50_'+str(n)+'_'+str(N-n)+'.txt', 'w')
    
        Lines = line(n, N-n)

        for x in Lines:
            f.write(x)
            f.write('\n')
        f.close()
    
lines(50)
\end{lstlisting}
    
\end{codeblock}


\bibliographystyle{plain}
\bibliography{references}

\end{document}

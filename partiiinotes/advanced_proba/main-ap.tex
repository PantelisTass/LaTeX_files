\documentclass{article}
% preamble.tex
\usepackage{amsmath, amsthm, amssymb}
\usepackage{mathtools}
\usepackage{mathabx}
\usepackage{adjustbox}
\usepackage{graphicx}% Required for inserting images
\usepackage{tikz-cd} 
\usepackage{tikz}
\usepackage{comment}
\usetikzlibrary{calc, graphs, positioning, backgrounds, decorations.pathmorphing}
\usepackage{forest}
\usepackage{pgfplots, pgfplotstable}
\usepackage{indentfirst}
\pgfplotsset{compat=1.17}
\usepackage[a4paper,width=150mm,top=25mm,bottom=25mm]{geometry}
\usepackage{fancyhdr}
\pagestyle{fancy}
\usepackage{ifthen}
\usepackage{float}
\usepackage{subcaption}
\usepackage{xkeyval} % Load xkeyval first
\usepackage[colorlinks=true,linkcolor=blue]{hyperref} % Load hyperref with simplified options
\usepackage[dvipsnames]{xcolor}
\usepackage{tcolorbox}
\usepackage{wrapfig}
\usepackage[framemethod=tikz]{mdframed}
\usepackage{atbegshi}% http://ctan.org/pkg/atbegshi


\definecolor{block-gray}{gray}{0.95}
\newtcolorbox{examplesblock}[2][]{%
    grow to right by=-0mm,
    grow to left by=-0mm, 
    opacityback=0,
    frame hidden, boxrule=0pt,
    boxsep=0pt,
    breakable,
    enhanced jigsaw,
    borderline west={2pt}{0pt}{black},
    title={#2\par},
    colbacktitle={black},
    coltitle={black},
    fonttitle={\bfseries},
    attach title to upper={},
    #1,
}

\definecolor{}{gray}{0.95}
\newtcolorbox{codeblock}[2][]{%
    grow to right by=-0mm,
    grow to left by=-0mm, 
    opacityback=0,
    frame hidden, boxrule=0pt,
    boxsep=0pt,
    breakable,
    enhanced jigsaw,
    borderline west={2pt}{0pt}{black},
    title={#2\par},
    colbacktitle={black},
    coltitle={black},
    fonttitle={\large\bfseries},
    attach title to upper={},
}

\titleformat{\chapter}[display]   
{\normalfont\huge\bfseries}{\chaptertitlename\ \thechapter}{20pt}{\Huge}   
\titlespacing*{\chapter}{0pt}{-30pt}{40pt}

\newtheoremstyle{mystyle}%                % Name
  {1pt}%                                     % Space above
  {}%                                     % Space below
  {\normalfont}%                                     % Body font
  {}%                                     % Indent amount
  {\bfseries}%                            % Theorem head font
  {.}%                                    % Punctuation after theorem head
  { }%                                    % Space after theorem head, ' ', or \newline
  {\thmname{#1}\thmnumber{ #2}\thmnote{ (#3)}}%                                     % Theorem head spec (can be left empty, meaning `normal')

  \theoremstyle{mystyle}
\newtheorem*{remark}{Remark}

\NewDocumentCommand{\newframedtheorem}{O{}momo}{%
  \IfNoValueTF{#3}
   {%
    \IfNoValueTF{#5}
     {\newtheorem{#2}{#4}}
     {\newtheorem{#2}{#4}[#5]}%
    }
   {\newfratheorem{#2}[#3]{#4}}
  \tcolorboxenvironment{#2}{#1}%
}
\newmdtheoremenv{theorem}{Theorem}[section]


\newtheorem{corollary}{Corollary}[section]
\newmdtheoremenv{boxcor}{Corollary}[theorem]
\newtheorem{lemma}[theorem]{Lemma}
\newtheorem{prop}[section]{Proposition}
\newtheorem{definition}{Definition}[section] 
%\newtheorem{theorem}{Theorem}[section] 

\newmdtheoremenv{boxdef}{Definition}[section]
\newmdtheoremenv{boxlemma}{Lemma}[section]
\newmdtheoremenv{boxprop}{Proposition}[section]

\newcommand{\here}[2]{\tikz[remember picture]{\node[inner sep=0](#2){#1}}}

\newcommand{\N}{\ensuremath{\mathbb{N}}}
\newcommand{\R}{\ensuremath{\mathbb{R}}}
\newcommand{\Z}{\ensuremath{\mathbb{Z}}}
\newcommand{\Q}{\ensuremath{\mathbb{Q}}}
\newcommand{\C}{\ensuremath{\mathbb{C}}}
\newcommand{\PP}{\ensuremath{\mathbb{P}}}
\newcommand{\F}{\mathcal{F}}

\newcommand{\Real}{\operatorname{Re}}
\newcommand{\Img}{\operatorname{Im}}

\DeclareMathOperator{\spn}{span}
\DeclareMathOperator{\Id}{Id}
\DeclareMathOperator{\essup}{essup}
\DeclareMathOperator{\scrC}{\mathcal{C}}
\DeclareMathOperator{\scrM}{\mathcal{M}}
\DeclareMathOperator{\supp}{supp}
\DeclareMathOperator{\intr}{int}
\DeclareMathOperator{\conv}{conv}
\DeclareMathOperator{\diam}{diam}
\DeclareMathOperator{\ext}{Ext}
\DeclareMathOperator{\dist}{dist}


\newcommand{\inmath}[2]{$#1\in#2$}
\newcommand{\allin}[2]{for all $#1\in#2$}
\newcommand{\isthere}{there exists }

\newcommand{\vectsp}[1]{\mathbf{#1}}
\newcommand{\norm}[1]{\left\lVert#1\right\rVert}
\newcommand{\unitball}[1]{B_{#1}}
\newcommand{\intunitball}[1]{D_{#1}}
\newcommand{\unitsphere}[1]{S_{#1}}

\newcommand{\bigslant}[2]{{\raisebox{.2em}{$#1$}\left/\raisebox{-.2em}{$#2$}\right.}}
\newcommand{\bracket}[2]{\langle #1, #2\rangle}
 % Method of Enrico Gregorio ("egreg")
 % https://tex.stackexchange.com/a/22255/13492:
 \newcommand\restrict[2]{% make the whole thing an ordinary symbol
   \left.\kern-\nulldelimiterspace % automatically resize the bar with \right
   #1% the function
   \littletaller % pretend it's a little taller at normal size
   \right|_{#2}%
   }
 \newcommand{\littletaller}{\mathchoice{\vphantom{\big|}}{}{}{}}

 % Method of Heiko Oberdiek
 % https://tex.stackexchange.com/a/67233/13492
 \makeatletter
 \newcommand*{\scaleddelims}[3]{%
   \ensuremath{%
     \mathpalette{\@scaleddelims{#1}{#2}}{#3}%
   }%
 }   
 \newcommand*{\@scaleddelims}[4]{%
   % #1: left delimiter
   % #2: right delimiter
   % #3: \displaystyle, \textstyle, ...
   % #4: inner formula
   \begingroup
     #3%
     \sbox0{$\m@th#3\vphantom{A}#4$}%
     \setbox2\vbox{\hbox{$\m@th#3#1$}\kern\z@}%
     \setbox4\vbox{\hbox{$\m@th#3#2$}\kern\z@}%
     \setbox6\hbox{$#3\vcenter{}$}%
     \ifx\downharpoonleft#1\relax  
       \let\DelimLeft=L%
     \else\ifx\upharpoonleft#1%
       \let\DelimLeft=L%
     \else\ifx\downharpoonright#1%
       \let\DelimLeft=R%
     \else\ifx\upharpoonright#1%
       \let\DelimLeft=R%
     \fi\fi\fi\fi
     \ifx\downharpoonleft#2\relax
       \let\DelimRight=L%
     \else\ifx\upharpoonleft#2\relax
       \let\DelimRight=L%
     \else\ifx\downharpoonright#2\relax
       \let\DelimRight=R%
     \else\ifx\upharpoonright#2\relax
       \let\DelimRight=R%
     \fi\fi\fi\fi
     \ifx\DelimLeft L%
       \wd2=.6\wd2
     \fi
     \ifx\DelimRight L%
       \wd4=.6\wd4
     \fi
     \ifx\DelimLeft R%
       \sbox2{\kern-.4\wd2\box2}%
     \fi
     \ifx\DelimRight R%
       \sbox4{\kern-.4\wd4\box4}%
     \fi
     \dimen0=\ht0 %
     \advance\dimen0 by -\ht6 %
     \dimen2=\dp0 %
     \advance\dimen2 by \ht6 %
     \ifdim\dimen2>\dimen0 %  
       \dimen0=\dimen2 %
     \else
       \dimen0=\dimen0 %
     \fi
     \dimen2=\ht6 %
     \advance\dimen2 by -\dimen0 %
     \dimen0=2\dimen0 %
     \def\DelimCorr{%  
       \mskip.5\thinmuskip
       \nonscript\mskip.5\thinmuskip
     }%
     \mathopen{%
       \ifx\DelimLeft R\DelimCorr\fi
       \raisebox{\dimen2}{\resizebox{!}{\dimen0}{\box2}}%
       \ifx\DelimLeft L\DelimCorr\fi
     }%
     \begingroup
       #3#4%
     \endgroup
     \mathclose{%
       \ifx\DelimRight R\DelimCorr\fi
       \raisebox{\dimen2}{\resizebox{!}{\dimen0}{\box4}}%
       \ifx\DelimRight L\DelimCorr\fi
     }%
   \endgroup
 }\makeatother
 %

 \newcommand{\restr}[2]{#1\scaleddelims{\kern-0.5\nulldelimiterspace\upharpoonright}{\vphantom{.}}{_{#2}}}
\newcommand{\diff}{\ensuremath{\operatorname{d}\!}}

  % Include the preamble from an external file

%\AtBeginDocument{\AtBeginShipoutNext{\AtBeginShipoutDiscard}\addtocounter{page}{-1}}
\fancyhead{}
\fancyhead[L]{\text{Advanced Probability}}
\fancyhead[R]{\text{Pantelis Tassopoulos}}

\title{\Huge Part III Advanced Probability \\ 
\huge Based on lectures by P. Sousi}
\author{\Large Notes taken by Pantelis Tassopoulos}
\date{\Large Michaelmas 2023}

\begin{document}

\maketitle
\newpage

\tableofcontents
\newpage 

\section{Conditional Expectation}
\mymark{Lecture 1}\subsection{Basic definitions}
Let $ (\Omega, \F, \PP) $ be a probability space. Remember the following definitions 
\begin{boxdef}[Sigma algebra]\label{def: sigma algebra}
	$ \F $ is a sigma algebra if and only if: ($ \F\in \mathcal{P}{\Omega} $)
	\begin{enumerate}
		\item $ \Omega \in \F $
		\item $ A\in \F \implies A^c\in \F$
		\item $ (A_n)_{n\in \N}\subseteq \F \implies \displaystyle\bigcup_{n\in\N}A_n\in\F $
	\end{enumerate}
\end{boxdef}
\begin{boxdef}[Probability measure]\label{def: prob measure}
	$ \PP $ is a probability measure if
	\begin{enumerate}
		\item $ \PP:\F\to [0,1] $ (i.e. a set function)
		\item $ \PP(\Omega) = 1 $, and $ \PP(\emptyset) = 0 $
		\item $ (A_n)_{n\in \N} $ pairwise disjoint $ \implies \PP\left(\displaystyle\bigcup_{n\in\N}A_n\right) =\displaystyle\sum^{\infty}_{n=1}\PP(A_n)$.
	\end{enumerate}
	
\end{boxdef} 
\begin{boxdef}[Random Variable]\label{def: rv}
	$ X:\Omega\to \R $ is a \underline{random variable} if for all $ B $ open in $ \R $, $ X^{-1}(B)\in \F $.
\end{boxdef}
\begin{remark}
	Observe that the sigma algebra $ \mathcal{G}=\{B\subseteq\R: X(B)\in\F\}\supseteq \mathcal{O} \implies \mathcal{G}\supseteq \mathcal{B}(\R) $, the former being the collection of open sets in $ \R $ and the latter the Borel sigma algebra on $ \R $ with the usual topology, namely, $ \sigma(\mathcal{O})$ (see below for the notation).
\end{remark}

Let $ \mathcal{A} $ be a collection of subsets of $ \Omega $. We define 
\[\begin{array}{ll}
	\sigma(\mathcal{A}) &= \text{smallest sigma algebra containing $ \mathcal{A} $} \\
     &=\displaystyle \bigcap \{\mathcal{T}:\mathcal{T} \text{ sigma algebra containing }\mathcal{A}\}.
\end{array}
\]

\begin{boxdef}[Borel sigma algebra on $ \R $]\label{def: borel sigma alg}
	Let $ \mathcal{O} = \{\text{open sets} \R\} $. Then, the Borel sigma algebra $ \mathcal{B}(\R)( \coloneq \mathcal{B} ) $ is defined as above, namely, 
	\[\mathcal{B}(\R)\coloneq \sigma(\mathcal{O}).\]
\end{boxdef}

Let $ (X_i)_{i\in I} $ be a family of random variables, then $ \sigma(X_{i}:i\in I) =$ the smallest sigma algebra that makes them all measurable. We also have the characterisation 
$ \sigma(X_{i}: i\in I) = \sigma(\{\underbrace{\{\omega\in \Omega: X_{i}(\omega)\in B\}}_{X^{-1}_{i}(B)}, i\in I, B\in \mathcal{B}(\R)\})$.

\subsection{Expectation}

Note we use the following for the indicator function on some event $ A $
\[\begin{array}{ll}
	\mathbf{1}(A)(x) &= \quad \mathbf{1}(x\in A)\\ 
     &\coloneqq \left. \begin{array}{cc}
    &1, \quad x\in A \\
    &0, \quad x\notin A
\end{array}\right\}, \quad A\in \F.
\end{array}
\]


We now begin the construction of the expectation of generic random variables.\\

\underline{Positive simple random variables:} $X = \displaystyle\sum^{
n}_{i=1}\mathbf{1}(A_{i}), c_{i}\geq 0, A_{i}\in\F.  $.
\[
	\mathbb{E}[X]\coloneqq\displaystyle\sum^{n}_{i=1}c_{i}\PP(A_{i}). 
\]


\underline{Non-negative random variables:} $ (X\geq 0). $
We proceed by approximation. Namely, let $ X_{n}(\omega)\coloneqq 2^{-n}\lfloor 2^{-n}\cdot X(\omega)\rfloor \land n \uparrow X(\omega) , n\to \infty$. Now, by monotone convergence, 
\[
	\mathbb{E}[X]\coloneqq \uparrow \displaystyle\lim_{n\to\infty}\mathbb{E}[X_{n}]=\displaystyle\sup\mathbb{E}[X].
\]

\underline{General random variables:} Have the decomposition $ X = X^{+}-X^{-} $, where $ X^{+} = X\lor 0, X^{-}=-X\land 0 $. If one of $ \mathbb{E}[X^{+}], \mathbb{E}[X^{-}]  <\infty $ then set 
\[
	\mathbb{E}[X]\coloneqq \mathbb{E}[X^{+}]-\mathbb{E}[X^{-}].  
\]

\begin{boxdef}\label{def: integrable rv}
	$ X $ is called \underline{integrable} if $ \mathbb{E}[|X|]<\infty $.
\end{boxdef}

\begin{boxdef}\label{def: cond prob event}
Let $ B\in \F $ with $ \PP(B)>0 $. Then for all $ A\in \F $, set 
\[
\PP(A|B)\coloneqq \frac{\PP(A\cap B)}{\PP(B)}
\]

Now for an integer-valued random variable $ X $, we set:
\[
	\mathbb{E}[X|B]\coloneqq \frac{\mathbb{E}[X\cdot \mathbf{1}_{B}]}{\PP(B)}
\]

\end{boxdef}

end{document}
